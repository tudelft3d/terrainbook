%!TEX root = ../terrainbook.tex
% chktex-file 46

\setchapterpreamble[u]{\margintoc}
\graphicspath{{spacedem/}}


\chapter{Space-based terrains}% or global terrains
\label{chap:spacedem}

We define as ``space-based terrains'' the datasets that cover (most of) the Earth.
They are also often referred to as \emph{global DEMs}.
Those datasets require different acquisition methods from the local ones, since flying an airplane or performing local surveys at the scale of the Earth is not really feasible (or is it?).
The acquisition instruments used must be space-borne, \ie\ mounted on a satellite for instance.


gDEMs are useful in several applications, especially for environmental studies such as geological studies, hydrological modelling, ecosystems dynamics.

gDEMS are several properties and characteristics that apply only to them, and we report in this chapter on the main ones.
% We first describe global acquisition techniques, then we discuss the properties (and errors and biases, etc.) that the datasets collected will have, 


%%%%%%%%%%%%%%%%%%%%
%
\section{Acquisition of gDEMs}

\begin{itemize}
  \item InSAR
  \item Space lidar (ICESat-2 + GEDI)
\end{itemize}


%%%%%%%%%%%%%%%%%%%%
%
\section{Characteristics of gDEMs}

\begin{itemize}
  \item CRS
  \item DSM (more than DTM)
  \item size of datasets
  \item resolution (often in degrees)
  \item errors
  \item integration with the sea-level datasets
\end{itemize}


%%%%%%%%%%%%%%%%%%%%
%
\section[Most common products]{Most common products available}

\begin{itemize}
  \item STRM
  \item CopernicusDEM
  \item ICESat-2 (the gridded version?)
  \item GEDI (the gridded version?)
  \item FABDEM
\end{itemize}


%%%%%%%%%%%%%%%%%%%%
%
\section{Conversion DSM to DTM}

Some examples of how done?

%%%%%%%%%%%%%%%%%%%%
%
\section{Notes \& comments}


%%%%%%%%%%%%%%%%%%%%
%
\section{Exercises}

\begin{enumerate}
  \item What is what?
\end{enumerate}
