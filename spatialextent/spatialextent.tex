%!TEX root = ../terrainbook.tex

\graphicspath{{spatialextent/}}

\chapter{The spatial extent of a point cloud}
\label{chap:spatialextent}

\begin{figure}[h]
  \centering
  \begin{subfigure}[b]{0.22\linewidth}
    \centering
    \includegraphics[page=1,width=\textwidth]{figs/idea.pdf}
    \caption{A set of points in $\mathbb{R}^2$}
  \end{subfigure}%
  \quad
  \begin{subfigure}[b]{0.22\linewidth}
    \centering
    \includegraphics[page=2,width=\textwidth]{figs/idea.pdf}
    \caption{Its convex hull}
  \end{subfigure}
  \quad
  \begin{subfigure}[b]{0.22\linewidth}
    \centering
    \includegraphics[page=3,width=\textwidth]{figs/idea.pdf}
    \caption{A $\chi$-shape}
  \end{subfigure}%
  \quad
  \begin{subfigure}[b]{0.22\linewidth}
    \centering
    \includegraphics[page=4,width=\textwidth]{figs/idea.pdf}
    \caption{An $\alpha$-shape}
  \end{subfigure}
\caption{Different methods to obtain the spatial extent of a given set of points in the plane.}
\label{fig:ideas}  
\end{figure}

Given a point cloud, one operation that practitioners need to perform is to define the spatial extent of the dataset.
That is, they need to define the region, which is represented by a one or more polygons, that best abstracts the set of points; in the context of terrains this region is often in two dimensions.
This is useful for instance to calculate the area covered by a dataset, to convert it to other formats (\eg\ raster), to get a quick overview of several datasets it is faster to load a few polygons and not billions of points, etc.

%

The spatial extent is often called: envelope, hull, concave hull, or footprints.
It is important to notice that the spatial extent is not uniquely defined, as Figure~\ref{fig:ideas} shows there are several potential regions for a rather simple set of points, and most of these could be considered `correct' by a human.

%

We present in this chapter a few methods/algorithms that are used in practice to define the spatial extent of a set of points in $\mathbb{R}^2$, which implies that the points in a point cloud are projected to the plane.



%%%%%%%%%%%%%%%%%%%%
%
\section{Notes \& comments}



%%%%%%%%%%%%%%%%%%%%
%
\section{Exercises}

\begin{enumerate}
  \item When converting isolines to a TIN, what main ``problem'' should you be aware of? Describe \emph{in details} one algorithm to convert isolines (given for instance in a \emph{shapefile}) to a TIN and avoid this problem.
  \item How would the isocontours of a 2.75D terrain look like?
  \item In Section~\ref{sec:structuring}, it is mentioned that merging the segments will form on polygon. But how to ensure that the orientation of that resulting curve is consistent, that it is for instance having higher terrains on the right?
  \item Given a raster terrain (GeoTiff format) that contains several cells with \texttt{no\_data} values, describe the methodology you would use to extract contour lines from it. As a reminder, contours lines should be closed curves, except at the boundary of the dataset.
\end{enumerate}