%!TEX root = ../terrainbook.tex
% chktex-file 46

\setchapterpreamble[u]{\margintoc}


\chapter{Visibility queries on terrains}%
\label{chap:visibility}

\graphicspath{{visibility/figs}}

Several applications using terrains involve \emph{visibility queries}, \ie\ given a viewpoint, which area of the surrounding terrain is visible?
Examples of such applications are many: optimal position of telecommunication towers, path planning for hiking (to ensure the nicest views), estimation of the view for scenic drives, estimation of visual damage when trees in a forest are cut, etc.
There are also several related problems.
Two closely related examples are the estimation of shadows (position of the sun continually varies, also with seasons) and the calculation of the solar irradiance (how much sun light will a certain area have per day/week).
These can be used to estimate the photovoltaic potential (where can we best install solar panels?), for estimating where snow will accumulate in mountains, or for estimating the temperature of the ground (necessary for climate modelling), among other applications.

%

When referring to visibility problems, we address the following two fundamental problems:
\begin{description}
  \begin{marginfigure}
    \centering
    \includegraphics[width=\linewidth]{overview_los}
    \caption{Line-of-sight between $v$ and $q$; $q$ is not visible.}%
    \label{fig:overview_los}
  \end{marginfigure} 
  \item[line-of-sight (LoS):] given a viewpoint $v$ and another point $q$, does $v$ sees $q$ (and vice-versa)? Or, in other words, does the segment $vq$ intersects the terrain? The result is either True or False.
  \begin{marginfigure}
    \centering
    \includegraphics[width=\linewidth]{overview_viewshed}
    \caption{The viewshed at the location marked with a red star (green = visible; maximum view distance (dark grey) is set to \qty{15}{\km}).}%
    \label{fig:overview_viewshed}
  \end{marginfigure} 
  \item[viewshed:] given a viewpoint $v$, which area of the surrounding terrain is visible? The result is a polygon (potentially disconnected) showing the locations and extent of what is visible from $v$. Usually the extent is limited to a certain ``horizon'', or radius of visibility. If the terrain is formed of different objects (\eg\ buildings), an object is either visible or not (simple case), or parts of objects can be visible (more complex).
\end{description}
Observe that for both problems, the viewpoint can either be directly on the terrain (at relative elevation \qty{0}{\m}) or at a given height (\qty{2}{\m} for a human, or \qty{30}{\m} for a telecommunication tower).

%

We discuss in this chapter the general problem of visibility as defined in computer graphics, and then discuss how terrains, being 2.5D surfaces, simplify the problem.
We discuss how to solve these problems for both TINs and grids.
% , and then we briefly explain how point clouds can be directly used for such queries.


%%%%%%%%%%%%%%%%%%%%
%
\section{Rendering + ray casting}

Rendering is the process of generating images from 2D or 3D scenes.
As shown in Figure~\ref{fig:Ray_trace_diagram}, it involves projecting the (3D) objects in a scene to an image (say 800$\times$800 pixels) and assigning one colour to each pixel.
\begin{marginfigure}
  \centering
  \includegraphics[width=\linewidth]{Ray_trace_diagram.pdf}
  \caption{Ray tracing builds the image pixel by pixel by extending rays into the scene. (Figure from \url{https://commons.wikimedia.org/wiki/File:Ray_trace_diagram.svg})}%
  \label{fig:Ray_trace_diagram}
\end{marginfigure}
In the simplest case the colour assigned is that of the closest object, but to obtain photorealistic images, lighting, shading, and other physics-based functions are often applied (however this goes beyond the scope of this book).

%

\emph{Ray casting} is used for each pixel: a ray is defined between the viewpoint $v$ and the centre of the pixel, and the closest object in the scene must be found.%
\index{ray casting}\marginnote{ray casting}
The main issue involves finding that closest object, and especially discarding the other objects lying behind it (an operation usually called hidden-surface determination/removal).%
\index{hidden-surface determination}\marginnote{hidden-surface determination}
Figure~\ref{fig:zbuffer} shows the idea for 2 objects ($O_1$ and $O_2$).
Observe that objects can \emph{partially} be hidden by others, and that the value of the pixel should always contain the closest object at that location.
\begin{marginfigure}
  \centering
  \includegraphics[width=\linewidth]{zbuffer.pdf}
  \caption{Two planar objects $O_1$ and $O_2$ are partially overlapping when viewed from $v$.}%
  \label{fig:zbuffer}
\end{marginfigure}

%

One often used algorithm to solving the hidden-surface determination is the \emph{depth-sort method}.
Its main idea is to define a coordinate reference system with $x$ and $y$ on the viewing plane, and $z$ perpendicular to it.
The objects are first sorted according to their maximal $z$-values, and the objects are drawn on the viewing plane from the furthest to the closest.
The value of a given pixel could therefore be redrawn several times, but its value will contain the colour of the closest object.
The algorithm assumes that all objects are planar polygons, which is for the case of terrains not an issue.
Observe that this algorithm is often referred to as the \emph{painter's algorithm} since it mimics the way one would draw a scene: details in the foreground are drawn ``over'' the background.

%

It should be noticed that it is possible that objects cannot be strictly $z$-ordered since their $z$-values can overlap.
Figure~\ref{fig:depthsort_issues} shows one example: the object $O_2$ from Figure~\ref{fig:zbuffer} was slightly rotated and now part of it is in front of $O_1$ and part of it is behind.
\begin{marginfigure}
  \centering
  \includegraphics[width=\linewidth]{depthsort_issues.pdf}
  \caption{Part of $O_2$ is behind $O_1$ and part is in front.}%
\label{fig:depthsort_issues}
\end{marginfigure}
The solution to this is to decompose one of the objects by the plane of the other, and to process all the parts as different objects.

%%

An efficient implementation of the depth-sort algorithm requires indexing the objects in the scene, and for this BSP-trees are commonly used.
\marginnote{BSP-tree: binary space partitioning}
A depth order for the scene can now be obtained by a traversal of the BSP tree.



%%%%%%%%%%%%%%%%%%%%
%
\section{For 2.5D terrains, the problem is simpler}[2.5D terrains are simple]

% Simplifying the problem by sorting the triangles
The problem is simplified for terrains we can sort the cells (pixels or triangles) and, because a terrain is a 2.5D surface, and we can convert the problem to a 2D one.
We can then exploit the connectivity and adjacency between the 2D cells forming a terrain to minimise the number of objects to test (for intersections and for hidden-surface determination).


%%%
%
\subsection{Visibility in grids}

Solving visibility queries in grids is simpler than with triangles since the topology of the grid is implied (we have direct access to the neighbours of a given cell), and because grid cells are usually small we can assume that a grid cell is visible (or not) if its centre is visible (or not).
The same assumption is tricky for triangles, since these can be large; in practice it is often assumed that a triangle is visible if its 3 vertices are visible, but this varies per implementation.

%

We describe here how both LoS and viewshed queries can be implemented for grids; the same principles could be applied to TINs with minor modifications.


%%%
\paragraph{Line-of-sight.}
A LoS query, between a viewpoint $v$ and another point $q$, implies reconstructing the profile of the terrain along the vertical projection of $vq$ (let us call it $vq_{xy}$).
It then suffices to follow $vq$ and verify whether the elevation at any ($x,y$) location along the profile is higher than that of $vq$.
As shown in Figure~\ref{fig:los}, 
\begin{figure}
  \centering
  \includegraphics[width=\linewidth]{los}
  \caption{Line-of-sight for between $v$ and $q$. Observe that along the profile, the points with elevation are not equally spaced.}%
\label{fig:los}
\end{figure}
since the terrain is discretised into grid cells, there are 2 options to reconstruct the profile between $v$ and $q$:
\begin{enumerate}
  \item identify all the cells intersected by $vq_{xy}$, and assign the centre of each cell by projecting it to the terrain profile. This is what is done in Figure~\ref{fig:los}.
  \item consider the edges of the cells, collect all the edges that are intersected by $vq_{xy}$, and linearly interpolate the elevations. This is far more expensive to compute, and therefore less used in practice.
\end{enumerate}

The algorithm is thus as follows.
Start at $v$, and for each pixel $c$ encountered along $vq_{xy}$, verify whether the elevation value of $vq$ at that location is higher than the elevation of $c$.
If it is, then continue to the next pixel; if not, then there is an intersection and thus the visibility is False.
If the pixel containing $q$ is reached without detecting an intersection, then the visibility is True.


%%%
\paragraph{Viewshed.}
As shown in Figure~\ref{fig:viewshed},
\begin{figure}
  \centering
  \includegraphics[width=\linewidth]{viewshed}
  \caption{Viewshed for the point $v$; the blue circle is the radius of the horizon (\qty{5000}{\m} in this case).}%
\label{fig:viewshed}
\end{figure}
computing the viewshed from a single viewpoint $v$ implies that the LoS between $v$ and the centre of each pixel in a given radius is tested. 
The result of a viewshed is a binary grid; in Figure~\ref{fig:overview_viewshed}, True/visible pixels are green, and False/invisible ones are dark grey.

%

While this brute-force approach will work, several redundant computations will be made, since several of the rays from $v$ will intersect the same grid cells.
Furthermore, depending on the resolution, the number of cells in a \qty{5}{\km} radius (a reasonable value where humans can see) can become \emph{very} large.
As an example, with the AHN3 gridded version (\qty{50}{\cm} resolution), this means roughly 400 million queries ($(\frac{5000 \times 2}{0.5})^2$).

%

One alternative solution is shown in Figure~\ref{fig:viewshed}b: it involves sampling the grid cells intersecting the border of the visible circle (obtaining several centres $q_i$), and computing the visibility of each of the cells along the line segment $vq_i$ as we `walk' from $v$.
Observe that, along $vq_i$, it is possible that a point far from $v$ is visible, while several closer points are not; Figure~\ref{fig:viewshed}c gives an example.

One solution involves using so-called \emph{tangents}.
The current tangent $t_{cur}$ is first initialised as a vector pointing downwards.
Then, starting at $v$, we walk along the ray $vq_i$, and for each cell intersected its elevation $z$ is compared to the elevation of $t_{cur}$ at that location.
If $z$ is lower, then the cell is invisible.
If $z$ is higher, then the cell is visible and $t_{cur}$ is updated with a new tangent using the current elevation.

Viewsheds with several viewpoints $v_i$ are also very useful, think for instance of obtaining the viewshed along a road.
This can be computed by sampling the road at every \qty{50}{\m} and computing the viewsheds from each of the points. 
Each viewshed yields a binary grid, and it suffices to use a map algebra operator to combine the results into one grid (if one cell is visible from any viewpoint, then it is visible).


%%%
%
\subsection{Visibility in TINs}

Using the depth-sort algorithm for arbitrary triangles would require using a BSP-tree for indexing and sorting the triangles, and some triangles would need to be decomposed, as explained above.
Figure~\ref{fig:acyclicity} shows one simple example.
\begin{marginfigure}
  \centering
  \includegraphics[width=\linewidth]{acyclicity.pdf}
  \caption{The 3 triangles $\tau_1$, $\tau_2$, and $\tau_3$ form a cycle when viewed from the viewpoint $v$, and it is not possible to sort them from furthest to closest (without decomposing them).}%
  \label{fig:acyclicity}
\end{marginfigure}

%

However, it has been proven that Delaunay triangulations are \emph{acyclic} for any fixed viewpoint. 
In other words, the in-front/behind relationship for the triangles of a DT, with respect to a given viewpoint, is acyclic (see Figure~\ref{fig:ordering_triangles}).
\begin{marginfigure}
  \centering
  \includegraphics[width=\linewidth]{ordering_triangles.pdf}
  \caption{The triangles in a DT can be ordered in an in-front/behind manner when viewed from a viewpoint.}%
  \label{fig:ordering_triangles}
\end{marginfigure}
Therefore, to obtain the triangles intersecting a ray coming out of a viewpoint (ordered from the closest to farthest), it suffices to modify slight the point location algorithm from Section~\ref{sec:dtwalk}.
This operation can be performed in 2D, by projecting the triangles of the TIN to the $xy$-plane.

%

This means that visibility queries in TINs---like in grids---are greatly simplified compared to the general case where the ordering of objects is the main difficulty (and handling overlapping objects like in Figure~\ref{fig:depthsort_issues}).



%%%%%%%%%%%%%%%%%%%%
%
\section{Notes and comments}

The `tangent algorithm' to compute viewsheds was first described by \citet{Blelloch90}.

The description here is inspired by that of \citet{DeFloriani99-1}.

\citet{Newell72} first proposed the depth-sorting algorithm and the decomposition necessary when polygons in the scene cannot be sorted from furthest to closest.

\citet{Edelsbrunner90} proved that Delaunay triangulations, in any dimensions, are acyclic.

%%%%%%%%%%%%%%%%%%%%
%
\section{Exercises}

\begin{enumerate}
  \item Explain why the spacing in Figure~\ref{fig:los}c along the profile has points that are not equally spaced.
  \item Your are given a 2.75D terrain of an area, it is composed of triangles, and your aim is to perform line-of-sight queries between some locations. Describe the algorithm that you will implement to perform the queries.
\end{enumerate}
