%!TEX root = ../terrainbook.tex

% About this book
% from a course, bundle of lecture notes
% link to videos
% acknowledgements


\chapter*{Preface}

This book presents an overview of algorithms and methodologies to reconstruct, manipulate, and extract information from terrains.

It covers different representations of terrains (\eg\ TINs, rasters, point clouds, contour lines) and presents techniques to handle large datasets.

% DTM are often only grid and TINS
% Modelling of terrains is one aspect of GIS that significantly changed with the arrival of new acquisition technologies such as airborne laser scanners and radar (SRTM), and yet books are often written years ago.


\paragraph*{Open material.}
This book is the bundle of the lecture notes that we wrote for the course \emph{Digitial terrain modelling} (GEO1015) in the MSc Geomatics at the Delft University of Technologies in the Netherlands.
The course is tailored for MSc students who have already followed an introductory course in GIS and in programming (in Python).

Each chapter is a lesson in the course, and each lesson is accompanied by a video introducing the key ideas and/or explaining some parts of the lessons.
All the videos are freely available online on the website of the course: \url{https://3d.bk.tudelft.nl/courses/geo1015/}


\paragraph*{Who is this book for?}
The book is written for students in Geomatics at the MSc level, but we believe it can be also used for at the BSc level.

Prerequisites are: GIS, background in linear algebra, programming course at the introductory level.


\paragraph*{Acknowledgements.}
We thank Balázs Dukai for help in proof-reading, and all the students of the first year of the course (2018--2019) who helped by pointing out errors and typos.
Also, the following students of the course all made pull requests to fix errors/typos: Chen Zhaiyu, Ardavan Vameghi, Li Xiaoai.






