%!TEX root = ../terrainbook.tex

\chapter*{Preface}

This book presents an overview of algorithms and methodologies to reconstruct terrains, to manipulate them, and to extract information from them.

It covers different representations of terrains (\eg\ TINs, rasters, point clouds, contour lines), discusses different applications (visibility analysis, runoff modelling, etc.), presents techniques to handle large datasets, and discusses related topics such as global elevation models and bathymetric datasets.

The book presents the theory and gives examples of libraries and software to perform certain tasks, but no code examples are provided.
We wanted the book to be language-agnostic.
The course website for which this book was developed (\url{https://3d.bk.tudelft.nl/courses/geo1015}) provides assignments where students use Python and/or C++.


\paragraph*{Open material.}
This book was primarily developed for the course \emph{Digital terrain modelling} (GEO1015) in the MSc Geomatics programme at Delft University of Technology in the Netherlands.
The course is tailored for MSc students who have already followed introductory courses in GIS, programming, and acquisition of geographical datasets.
Each chapter corresponds to a lesson in the course, whose content is also openly available: \url{https://3d.bk.tudelft.nl/courses/geo1015}


\paragraph*{Accompanying videos.}
Most of the chapters have a short video explaining the key concepts, and those are freely available online: \url{https://tudelft3d.github.io/terrainbook/videos}


\paragraph*{Who is this book for?}
The book is written for MSc Geomatics students, but it can also be used at the BSc level.
Prerequisites include: knowledge of GIS, background in linear algebra, and an introductory programming course.


\paragraph*{Acknowledgements.}
We thank Balázs Dukai for thoroughly proofreading the drafts of this book, and the many students of the GEO1015 course over the years who helped us by pointing---and often fixing with a pull request---the errors, typos, and weird sentences of this book. 
A special thanks to the students of the year 2018--2019 who had to deal with the first version of this book.
