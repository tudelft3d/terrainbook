%!TEX root = ../../terrainbook.tex
% chktex-file 46

\graphicspath{{appendices/normalplane/figs/}}

\chapter{Estimating the normals in a point cloud}%
\label{app:normalplane}

Given a point cloud $S$, the normal vector for a point $p \in S$ can be found by fitting a plane $P$ to the points in the local neighbourhood of $p$. 
The vector orthogonal to this plane is the normal vector $\vec{n}$.

In practice, we might want to use the 10 (or 15 or 20, depending on the resolution of $S$) nearest neighbours to $p$.
This can be efficiently (and easily with the many implementations available) performed with a $k$d-tree, see Section~\ref{sec:knn-m}.

To fit the plane, the preferred option is to use least-square fitting because it minimises the sum of squared distances between the points and the plane $P$.

We suggest to use Principal Component Analysis (PCA) to obtain the plane $P$, and to obtain its normal $\vec{n}$.
PCA allows us to identify the directions of maximum variance in a dataset, and it uses the \emph{eigenvalues} and \emph{eigenvectors} of the covariance matrix of a dataset.
\marginnote{eigenvalues and eigenvectors}%
\index{eigenvalues}\index{eigenvectors}
The eigenvector linked with the largest eigenvalue represents the direction where the variance is the largest, and the smallest eigenvalue where the variance is the smallest.

%

For our subset of 10 or 15 neighbouring points in $S$, the direction of maximum variance is the plane that best fits the data, and the normal vector is the direction of minimum variance.

% Observe that if you have points on the edge of a top of a roof then the normal will point somewhere else kinda.

It should also be noticed that the eigenvalues $\lambda_{1,2,3}$ (where $\lambda_1 \geq \lambda_2 \geq \lambda_3 \geq 0$) can be useful to calculate/estimate the local geometric properties such as the following:

\begin{equation}
\begin{aligned}
\textbf{linearity:} \quad L_\lambda &= \frac{\lambda_1 - \lambda_2}{\lambda_1} \\
\textbf{planarity:} \quad P_\lambda &= \frac{\lambda_2 - \lambda_3}{\lambda_1} \\
\textbf{sphericity:} \quad S_\lambda &= \frac{\lambda_3}{\lambda_1} \\
\end{aligned}
\end{equation}


